\chapter{Testování, nasazení a dokumentace}

% --------------------------------------------------------------------------------------------------
% Testování
% --------------------------------------------------------------------------------------------------

\section{Testování}

Testování je nedílnou součástí každé aplikace. Je nutné pro zajištění kvality kódu a správnosti poskytovaného produktu. Obecně se dá rozdělit na dva základní typy -- manuální, jenž spočívá v testování každé funkcionality a změny lidmi, a automatické, které využívá předem připravených programů nebo skriptů. Manuální testování oproti automatickému je přesnější a může poskytnout mnohem větší spektrum hodnocení, není však rychlé a je ovlivněno lidským faktorem. Automatické testování je proto preferováno vždy, když není potřeba mít zpětnou vazbu přímo od uživatele.

\todo{Daná kapitola bude dokončena po otestování aplikace uživateli}


\subsection{Automatické testování}

Obě části \gls{is} využívají automatického testování pro ověřování základních funkcionalit systému. Z hlediska výběru testovacího frameworku bylo vybíráno mezi Jest a Mocha. Po vyzkoušené integraci obou bylo rozhodnuto ponechat Jest, protože samostatně poskytoval veškerou potřebnou funkcionalitu. Mocha z tohoto hlediska potřebovala dodatečné knihovny.


\subsection{Manuální testování}

Pro testování uživatelského rozhraní nebylo vybráno automatické testování, protože nemůže poskytnout široké spektrum hodnocení. V této souvislosti skupina lidí byla požádána provést předem připravené scenáře a vyjádřit se k pohodlnosti rozhraní, rychlosti nalezení funkcionality a dalším kritériím hodnocení. 


První scénář:
\begin{olnar}
   \item Přihlaste se do informačního systému.
   \item Podívejte se na svoje upozornění.
   \item Zjistěte, kolik máte nových upozornění.
   \item \dots
\end{olnar}

Výsledky scénáře:

... 
% \begin{fig:table}
%    \begin{tabular}{| l c c |}
%       Subjekt & Výsledný čas & Subjektivní složitost (0-10) \\
%       \hline
%       Charlotte & 38 s & 3 \\
%       Alice & 42 s& 2 \\
%       John & 41 s & 3 \\
%       \hline
%    \end{tabular}
%    \caption{Výsledky prvního scénáře uživatelského testování}\label{test:results-1}
% \end{fig:table}

V závislosti na výsledcích byla provedena korekce rozhraní.


% --------------------------------------------------------------------------------------------------
% Nasazení
% --------------------------------------------------------------------------------------------------


\section{Nasazení}

Nasazení daného informačního systému je pouze dočasné a bude sloužit výhradně pro prezentační účely bakalářské práce. K tomuto účelu je využit osobní virtuální server s operačním systémem linux. Spuštění instancí databází je řízeno technologií kontejnerizace Docker. Příklad konfiguračního souboru se nachází v kořenovém adresáři serverové aplikace. Z hlediska zajištění bezpečnosti autorizace je na serveru povolen \gls{https} protokol s certifikátem podepsaným certifikační autoritou Let’s Encrypt. Po zajištění stabilního prostředí s detailnějším monitorováním procesů jsou aplikace spouštěny v pm2 -- démonu pro správu systému.

Daný případ konfigurace serveru není optimální. Pro dlouhodobý běh aplikace je nutné preciznější nastavení databází, omezení přístupu k serveru přes určité porty, zavedení automatického zálohování, systém oznámení o stavu aplikací a další náležitosti.


% --------------------------------------------------------------------------------------------------
% Dokumentace
% --------------------------------------------------------------------------------------------------


\section{Dokumentace}

Informační systém je dokumentován dvěma způsoby -- pomocí komentářů v samotném zdrojovém kódu a nezávislých příruček, jež obsahují všeobecné informace. Formát komentářů se řídí dle norem generátorů dokumentací. Pro serverovou aplikaci psanou v jazyce \texttt{typescript} se jedná o generátor TypeDoc, v případě webového klienta psaného v \texttt{javascript} jde o podobný generátor \texttt{JSDoc}. V případě potřeby existuje možnost vytvoření manuálu na základě okomentovaných částí.

Dodatečné příručky tvoří samostatné celky spravované frameworkem \texttt{docsify}. Uživatelská příručka popisuje jednotlivé případy užití na základě webového klienta. Vývojářská dokumentace zachycuje detaily, jež by bylo nevhodné dávat do zdrojového kódu -- popis inicializace aplikací, popis interakce komponentů, šablony pro vytváření určitých souborů a další náležitosti\footnote{Kopie obou příruček jsou dostupné na přiloženém médiu.}.
