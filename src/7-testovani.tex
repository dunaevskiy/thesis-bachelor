\chapter{Testování a nasazení}

\todo{Úvodní odstavec o testování}


% --------------------------------------------------------------------------------------------------
% Automatické testování
% --------------------------------------------------------------------------------------------------


\section{Automatické testování}

\todo{Porovnání JEST a Mocha, zmínka o podpůrných knihovnách chai, sinon, apod.}
\todo{Speciální configurace DB pro testování}

\subsection{Unit testy}
% Unit testy ověřují správnost jednotlivých metod.

\subsection{Integrační testy}


% --------------------------------------------------------------------------------------------------
% Manuální testování
% --------------------------------------------------------------------------------------------------


\section{Manuální testování}


Pro testování uživatelského rozhraní nebylo vybráno automatické testování, protože nemůže poskytnout široké spektrum hodnocení. V této souvislosti skupina lidí byla požádána provést předem připravené scenáře a vyjádřit se k pohodlnosti rozhraní, rychlosti nalezení funkcionality, ...

První scénář:
\begin{olnar}
   \item Přihlaste se do informačního systému.
   \item Podívejte se na svoje upozornění.
   \item Zjistěte, kolik máte nových upozornění.
   \item \dots
\end{olnar}

Výsledky scénáře:

% \begin{fig:table}
%    \begin{tabular}{| l c c |}
%       Subjekt & Výsledný čas & Subjektivní složitost (0-10) \\
%       \hline
%       Charlotte & 38 s & 3 \\
%       Alice & 42 s& 2 \\
%       John & 41 s & 3 \\
%       \hline
%    \end{tabular}
%    \caption{Výsledky prvního scénáře uživatelského testování}\label{test:results-1}
% \end{fig:table}

V závislosti na výsledcích byla provedena korekce rozhraní.


% --------------------------------------------------------------------------------------------------
% Konfigurace serveru
% --------------------------------------------------------------------------------------------------


\section{Konfigurace serveru}

\todo{SSL certifikát -- nastavení, důležitost}
\todo{nginx -- proxy,...}

\todo[color=colordiagram]{Deployment diagram}


% --------------------------------------------------------------------------------------------------
% Nasazení
% --------------------------------------------------------------------------------------------------


\section{Nasazení}
\todo[color=colordiagram]{Lifecycle}
\todo{pm2}
