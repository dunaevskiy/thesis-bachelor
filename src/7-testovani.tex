\chapter{Testování, nasazení a dokumentace}


% --------------------------------------------------------------------------------------------------
% Testování
% --------------------------------------------------------------------------------------------------


\section{Testování}

Testování je nedílnou součástí každé aplikace. Je nutné pro zajištění kvality kódu a správnosti poskytovaného produktu. Obecně se dá rozdělit na dva základní typy -- manuální, jež spočívá v testování každé funkcionality a změny lidmi, a automatické, které využívá předem připravených programů nebo skriptů. Manuální testování oproti automatickému je přesnější a může poskytnout mnohem větší spektrum hodnocení, není však rychlé a je ovlivněno lidským faktorem. Automatické testování je proto preferováno vždy, když není potřeba mít zpětnou vazbu přímo od uživatele.


\subsection{Automatické testování}

Obě části \gls{is} jsou připravené pro automatické testování pro ověřování základních funkcionalit systému. Z hlediska výběru testovacího frameworku bylo vybíráno mezi Jest a Mocha. Po vyzkoušené integraci obou bylo rozhodnuto ponechat Jest, protože samostatně poskytoval veškerou potřebnou funkcionalitu. Mocha z tohoto hlediska potřebovala dodatečné knihovny.


\subsection{Manuální testování}

Vzhledem ke komplexnosti a časové náročnosti projektu manuální testování poskytuje jedinou možnost precizního testování aplikace prostřednictvím akceptačních testů uvedených na přiloženém médiu. Je časově náročnější, ale poskytuje širší spektrum hodnocení. V této souvislosti skupina lidí byla požádána provést předem připravené scénáře a oznámit nalezené chyby či jiné problémy. V závislosti na výsledcích byla provedena korekce rozhraní. V budoucnu drtivá většina akceptačních testů má být nahrazena automatickými testy.


% --------------------------------------------------------------------------------------------------
% Nasazení
% --------------------------------------------------------------------------------------------------


\section{Nasazení}

Způsob nasazení daného informačního systému je popsán ve vývojářské dokumentaci. Pro nasazení \gls{is} je přepokládána dispozice virtuálního serveru s operačním systémem linux. Spuštění instancí databází je řízeno technologií kontejnerizace Docker. Příklad konfiguračního souboru se nachází v kořenovém adresáři serverové aplikace. Pro zajištění stabilního prostředí s detailnějším monitorováním procesů jsou aplikace spouštěny v pm2 -- démonu pro správu procesů.

Daný případ konfigurace serveru není optimální. Pro dlouhodobý běh aplikace je nutné preciznější nastavení databází, omezení přístupu k serveru přes určité porty, zavedení automatického zálohování, systém oznámení o stavu aplikací, povolení \gls{https} protokolu a další náležitosti.


% --------------------------------------------------------------------------------------------------
% Dokumentace
% --------------------------------------------------------------------------------------------------


\section{Dokumentace}

Informační systém je dokumentován dvěma způsoby -- pomocí komentářů v samotném zdrojovém kódu a nezávislých příruček, jež obsahují všeobecné informace. Formát komentářů se řídí dle norem generátorů dokumentací. Pro serverovou aplikaci psanou v jazyce TypeScript se jedná o generátor TypeDoc, v případě webového klienta psaného v JavaScript jde o podobný generátor \texttt{JSDoc}. V případě potřeby existuje možnost vytvoření manuálu na základě okomentovaných částí.

Dodatečné příručky tvoří samostatné celky spravované frameworkem \texttt{docsify}. Uživatelská příručka popisuje vybrané případy užití na základě webového klienta. Vývojářská dokumentace zachycuje detaily, jež by bylo nevhodné dávat do zdrojového kódu\footnote{Kopie obou příruček jsou dostupné na přiloženém médiu.}.
