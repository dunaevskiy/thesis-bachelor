\chapter{Rozvoj informačního systému}

Vzniklá implementace aplikace je prvním prototypem -- beta verze, jež potřebuje zčásti provést optimalizaci a následně otestovat v reálném nasazení se skutečnými uživateli, aby se mohly detekovat chyby, na které se nepřišlo během umělého testování (automatického i uživatelského).

Informační systém dle způsobů rozvoje lze rozdělit na dva hlavní směry -- obecné univerzální zdokonalování a rozvoj se zaměřením na \gls{fit} \gls{čvut}, na základě analýzy služeb kterého byl navržen daný \gls{is}. 

%Z hlediska obecného rozvoje lze definovat následující úkoly:

\begin{dl}   

   \item[DO00 Podpora internacionalizace a lokalizace]
   Aplikace v současném stavu podporuje pouze jeden jazyk -- angličtinu. Je potřeba vytvořit úložiště pro překládané výrazy, napsat implementaci middleware pro serverovou část klienta realizující detekování aktuálního jazyka uživatele (například s pomocí \gls{uri}) a přidat dosazování přeložených výrazů během \gls{ssr} i \gls{csr}. V případně nutnosti realizovat i jiné potřebné formátování.
    
   \item[DO01 Hromadné zakládání projektů] 
   Zakládání projektů se ne vždy omezuje založením jednoho dobrovolného projektu s jedním počátečním uživatelem v roli vedoucího. V případech zakládání projektů pro skupinu studentů nebo vyhlašováním otevřeného přihlašování na množinu semestrálních prací s otevřeným zadáním se nabízí možnosti hromadného zakládání projektů.

   Založit dobrovolné projekty hromadně -- definují se společné metainformace a počet kopií projektů. Výsledkem je skupina identických projektů s otevřeným přihlašováním.

   Založit povinné projekty hromadně -- definují se společné metainformace projektů a množina uživatelů. Výsledkem je množina založených identických projektů, pro každého uživatele jedna kopie. Pouze pro globální role s pokročilými možnostmi správy uživatelů.

   \item[DO02 Nové interpretátory částí obsahů a integrace se službami třetích stran]
   Interpretátory částí se vždy vážou na jeden určitý typ obsahu a jsou omezovány ukládáním informací pouze ve svém nebo serverovém úložišti. Pro rozšíření možností generování obsahu projektu je potřeba napsat víc typů interpretátorů (například zobrazování tabulek). V případně vhodných úprav životního cyklu rendrování obsahu je možné realizovat nahrávání či ukládání dat na vnější úložiště typu Google Drive, YouTube, Dropbox, Lucidchart a další.

   \item[DO03 Analýza využití systému a aktualizace \gls{ux}]
   V případě rozvoje sytému a zvyšování počtu uživatelů je třeba postupně zlepšovat i \gls{ux} klienta. To bude možné po implementaci skriptů pro sbírání statistik. Na základě jejich výsledků lze provést potřebné změny.
   
   \item[DO04 Optimalizace stávajícího systému]
   Vzhledem k nedokonalosti stávající implementace je potřeba provést optimalizaci některých funkcionalit. Mezi již známá slabá místa patří odesílání všech výsledků vyhledávání jedním souborem, absence automatického obnovování \gls{jwt} po vypršení doby platnosti, \dots
   
   \todo{Přidat další vzniklé implementační problémy}

   %V případě zaměření aplikace na \gls{čvut} \gls{fit} a integrace s existujícími službami lze systém doplnit o následující funkcionality:

   \item[FIT00 Integrace CVUT OAuth 2.0 serveru]
   Aplikace již počítá s integrováním jiných autorizačních serverů. V serverové části stačí přidat novou PassportJS strategii s vhodnou konfigurací pro přesměrování na autorizační server a realizovat mapování dat z odpovědi na databázi. Z hlediska uživatelského rozhraní je třeba přidat pouze jedno autorizační tlačítko. V případě potřeby využití \texttt{access token} autorizačního serveru bude potřeba rozšířit databázi a vyřešit ukládání a obnovování \texttt{access token} a \texttt{refresh token}.

   \item[FIT01 Integrace s FIT Klasifikace]
   Výsledné hodnocení projektů může být vhodné exportovat do služby FIT Klasifikace. V nejjednodušší verzi se jedná o zápis celkového počtu bodů projektu všem jeho členům registrovaným přes autorizační server \gls{čvut}. 

\end{dl}
