Cílem této práce byla analýza stávajících způsobů vedení studentských projektů zaměřená na \gls{fit} \gls{čvut} s následným návrhem a implementací \gls{oss} informačního systému. 

V souvislosti se stanoveným cílem první část této práce byla věnována rešerši některých dostupných systémů na \gls{fit} \gls{čvut} pro správu projektů. S pomocí slovního popisu a konceptuálních modelů byly zanalyzovány jejich hlavní rysy, poskytovaná funkcionalita a silné a slabé stránky. Následně na základě získaných znalostí a v souladu se zadáním bakalářské práce byla sepsána specifikace nového informačního systému pro podporu aktivit kolem studijních projektů. Vzniklá specifikace postupně popisuje hlavní životní cyklus projektů a jaké funkční a obecné požadavky budou kladeny na nový systém. V samostatných podkapitolách je popsána uživatelská činnost, primární aktivity jsou navíc zpřesněny konceptuálními modely v jazyku \gls{uml}.

V souladu s definovanou specifikací byl vytvořen obecný návrh \gls{oss} aplikace z hlediska vhodných technologií a architektury. Vzniklý systém se skládá ze dvou samostatných částí -- serverové aplikace poskytující veřejně dostupné \gls{api} a klientské webové aplikace přizpůsobené pro desktopové i mobilní rozhraní. Obě části jsou psány ve stejném skriptovacím jazyku JavaScript (případě TypeScript, který udržuje zpětnou kompatibilitu), to přináší přehlednost a jednotnost z hlediska využívaných knihoven. 

Serverová část disponuje vlastním uživatelským systémem založeném na vnějších autorizačních OAuth~2.0 serverech. Spolu s implementovaným systémem práv tvoří jednoduché rozhraní pro kontrolu přístupu uživatelů k jednotlivým funkcionalitám. Jako úložiště jsou využívány databáze PostgreSQL (udržování systémových entit a jejich relací), MongoDB (uchovávání obsahu projektů) a Redis (předávání autorizačních klíčů). Komunikace s dostupnými klientskými aplikacemi se provádí přes RESTful \gls{api}.

Implementace klienstkého webového rozhraní je založena na technologii \gls{spa} a \gls{ssr}. Z větší části představuje pouze vizualizaci dat ze serveru, ale speciální částí se stal způsob generování obsahu projektu. Logika generování vizuální reprezentace dat byla přenesena na GitHub úložiště, odkud je v případě potřeby automaticky stahována jednotlivými uživateli přes službu jsDelivr. 

Výběr všech technologií byl zdůvodněn v jednotlivých kapitolách práce. V případě nutnosti byly provedeny i některé srovnávací testy frameworků či implementačních řešení. K oběma aplikacím jsou dodávány testovací soubory a krátké dokumentace. V jedné z posledních kapitol je uveden seznam možných vylepšení informačního systému.

Nový informační systém je publikován na serveru GitHub, z důvodů nedostatku dat a zkušeností však není dokonalý. Hlavní cíl práce -- vytvoření nového \gls{oss} nástroje pro správu studentských projektů -- je splněn, sekundární cíl -- integrace se službami \gls{fit} \gls{čvut} -- nebyl plně realizován. Projekt z důvodu nedostatku potřebné funkcionality není hotový pro nabídnutí využití na \gls{fit} \gls{čvut}. Budoucí rozvoj a uplatnění služby budou dodatečně konzultovány s fakultou. V případě neúspěchu bude projekt nadále vyvíjen nezávisle.