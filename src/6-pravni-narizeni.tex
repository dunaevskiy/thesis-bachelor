\chapter{Právní náležitosti}

Daná kapitola zkoumá vliv legislativy na oblast webových aplikací, jaké nároky jsou kladeny v současné době a co je potřeba split v případě implementace \gls{is} pro akademické instituce. Zaměření kapitoly je pouze na Obecné nařízení na ochranu osobních údajů, ochrana autorských práv a další náležitosti nejsou předmětem analýzy.

Rozvoj oblasti webových služeb v posledních letech způsobil, že společnosti (například Facebook, Inc. a Twitter, Inc.) i jednotlivci působící na internetu sbíraly na vlastních serverech data dobrovolně poskytované jejich uživateli. Vznikla situace, kdy dané subjekty nebyly omezovány zákony a mohly do jisté míry nakládat se shromaždovanými daty dle jejich potřeb. Tuto situaci bylo potřeba řešit. Jedním z významných kroků, který změnil povinnosti subjektů (včetně majitelů webových služeb), jež shromažďují nebo zpracovávají osobní údaje občanů zemí \gls{eu} se stal den 25. května 2018, kdy nabylo účinnosti Obecné nařízení na ochranu osobních údajů neboli \gls{gdpr}~\cite{gdprDesc}. 

\gls{gdpr} přináší řadu povinností, dle článku~\cite{gdprPriprava} musí každá služba ovlivněná tímto nařízením poskytovat uživatelům práva pro přístup k osobním údajům v přehledné formě, dávat možnost opravy těchto údajů v případě nesprávnosti, případně odstranit všechny osobní údaje ze systémy. Za osobní údaje se v tomto případě považují všechny informace, na základě kterých lze identifikovat konkrétní osobu. Mohou mezi ně patřit konkrétní údaje (jméno, mobilní číslo) nebo skupina obecných (popis osoby)~\cite{gdprPrakticky}.

Informační sytém navrhovaný a implementovaný v dané bakalářské práci se snaží omezit množství uchovávaných osobních informací. Autorizace uživatelů probíhá přes autorizační servery třetích stran a uchovává pouze jejich unikátní \gls{id} v číselné podobě. Danou informaci může uživatel systému sám anonymizovat (pro uživatelé daný krok znamená výmaz účtu). Úplné odstraněné však z technického hlediska není možné, protože během existence účtu vznikají data, jež nepatří pouze uživateli (obsah projektů, tvorba a hodnocení iterací). Jsou nezbytně nutné pro plynulý běh \gls{is}. Nezávislá data typu upozornění jsou však odstraněna navždy. Jednotný přehled všech uživatelských informací je poskytován na stránce \gls{faq} informačního systému. Případné opravy údajů nejsou v danou chvíli zautomatizovány, budou řešeny osobně v komunikaci s poskytovatelem služby.

Sepsaní přehledu zpracování osobních údajů a podmínek ochrany osobních údajů, které jsou rovněž nutné pro poskytování webových služeb, jsou mimo rozsah dané práce, protože vyžadují pokročilejší právní znalosti.