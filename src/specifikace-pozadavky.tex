\section{Požadavky na systém}

Pro definování rozsahu práce slouží funkční a obecné požadavky. Funkční požadavky zachycují jednotlivé funkcionality vytvářené aplikace, jejich logickou stránku a chování. Obecné požadavky, neboli nefunkční, jsou požadavky, jež na systém kladou jistá omezení a tím upřesňují rozsah práce, avšak nedefinují žádnou aktivní složku (například uživatelská funkcionalita nebo serverová činnost). 


\subsection{Funkční požadavky}

\begin{dl}
   \item[FR00 Identita uživatelů]
   Součástí aplikace je uživatelský systém vyjma autorizačního serveru. Aplikace využívá pouze autorizační servery třetích stran, s jejich pomocí probíhá registrace i přihlášení. Identitu uživatele musí být možné přesně určit na všech webových stránkách, kromě skupiny stránek sloužící pro autentizaci či informování o základních náležitostech (například podmínky užití).

   \item[FR01 Globální role]
   Vsechny účastníci \gls{is} mají přidělenou jednu globální roli, která určuje jejich oprávnění v rámci aplikace. Ihned po registraci dostávají výchozí roli standardního uživatele. Existuje speciální role administrátor, jenž má neomezený přístup. Další role představují jiné množiny oprávnění, které jsou definovány v případech užití. U jednotlivých globálních rolí lze nastavit kritérium důvěryhodnosti. Uživatelé s danou rolí jsou v \gls{is} uvedeny se speciálním označením vedle uživatelského jména. Aplikace umožňuje měnit roli uživatele, přidávat nové role a měnit oprávnění.

   \item[FR02 Osobní informace uživatelů]
   Aplikace umožňuje účastníkům \gls{is} prohlížet si svoje osobní informace, které jsou uchovávány na serveru. Existuje funkcionalita pro odstranění bez možnosti obnovení (v případě nutnosti zachování určitých činností budou dané činnosti zanonymizovány).

   \item[FR03 Upozornění]
   Aplikace posílá účastníkům \gls{is} upozornění po každé významné činnosti (například ohodnocení práce, archivace projektu\dots{}), jež se jich přímo týká. Upozornění existují pouze v rámci \gls{is} a představují jednoduchou textovou zprávu. Účastníci mají možnost zobrazit svoje upozornění a označit je jako přečtené/nepřečtené nebo úplně odstranit.

   \item[FR04 Projekt -- Životní cyklus]
   Aplikace umožňuje účastníkům \gls{is} zakládat vlastní projekty, ve kterých se automaticky stávají vedoucími. Po založení je možné vytvořit tým, definovat cíle a začít tvořit obsah. V případě neaktuálnosti projektu se dá odstranit, tím zcela zanikne, nebo archivovat, pak existuje možnost ho kdykoliv obnovit. Archivací se zakáže veškerá činnost uživatelů na~daném projektu kromě rušení archivace a odchodu uživatelů z projektového týmu.

   \item[FR05 Projekt -- Správa dat]
   Každý projekt je definován množinou dat, jež ho popisují a prezentují. Data se dělí na vždy veřejné, skryté a volitelně skryté. Mezi veřejné patří: 

   \begin{ulnar}
      \item kategorie,
      \item název,
      \item datum založení,
      \item veřejný popis,
      \item seznam členů projektového týmu,
      \item tagy,
      \item status archivace.
   \end{ulnar}


   Skryté (dostupné pouze pro členy) jsou: 

   \begin{ulnar}
      \item seznam iterací a úkolů, 
      \item interní popis projektu, 
      \item snapshoty iterací a jejich hodnocení.
   \end{ulnar}
   

   Za volitelně skryté se považují vakantní pozice a samotný obsah.

   \item[FR06 Projekt -- Kategorie a tagy]
   Aplikace umožňuje třídění projektů do jednotlivých skupin s pomocí kategorií a tagů. Každý založený projekt povinně patří právě do jedné kategorie, která určuje jeho primární zaměření. Kategorie lze zakládat, upravovat a odstraňovat, pokud jsou prázdné. V případě nutnosti přesnější specifikace projektu je možné použít tagy, které budou primárně sloužit jako prostředek pro vyhledávání.

   \item[FR07 Projekt -- Tým a role]
   Na tvorbě obsahu projektu se podílí uživatelé s příslušným oprávněním. Pro přiřazování oprávnění slouží projektové role. Dle logického hlediska se dají rozdělit pouze na tři typy -- vedoucí, spolupracovník a návštěvník.

   \newpage
   Vedoucí mají neomezený přístup z hlediska práv. Prvním vedoucím projektu se vždy stává uživatel, jenž daný projekt založil. Každý vedoucí může prohlásit jiného člena týmu za vedoucího. Odejít z této role však lze pouze v případě, že~v projektu zůstane alespoň jeden vedoucí.

   Pravým opakem vedoucích jsou návštěvníci, kteří mají právo pouze prohlížet obsah, nikoliv do něj zasahovat. Volné přihlašování na roli návštěvníka lze zakázat, či povolit.

   Speciálním typem role je spolupracovník, který slouží pouze jako nadskupina pro množinu kapacitně omezených pracovních míst. Pracovní místa jsou definovány vedoucím pro inzerci náboru členů týmu. V případě, že je otevřené přihlašování, běžní uživatelé se mohou dobrovolně hlásit na vakantní pozice. Pokud vedoucí má status důvěryhodného uživatele, tak může přidat člena týmu i bez jeho souhlasu.

   \item[FR08 Projekt -- Vyhledávání]
   Aplikace umožňuje uživatelům vyhledávat existující projekty podle zadaných kritérií, pokud vedoucí projektu povolil jeho vyhledávání. Mezi filtry pro vyhledávání patří: 

   \begin{ulnar}
      \item stav archivace, 
      \item projekty, ve kterých je aktuální uživatel vedoucím, 
      \item projekty, do kterých má aktuální uživatel přístup, 
      \item dle kategorie, 
      \item dle tagů.
   \end{ulnar}
   

   \item[FR09 Projekt -- Iterace a úkoly]
   Aplikace umožňuje definovat plánovaný průběh projektu v podobě iterací a úkolů. Iterace je dlouhodobá, má název a plánované datum dokončení, slouží jako nadskupina pro jednotlivé úkoly. Úkoly se skládají z názvu, popisu, maximálního a minimálního počtu bodů. V průběhu tvoření obsahu projektu jednotlivé části obsahu označují úkoly jako splněné.

   \item[FR10 Projekt -- Správa obsahu]
   Aplikace umožňuje vedoucím a spolupracovníkům projektu tvořit jeho obsah. Ten může být soukromý, neboli dostupný pouze pro členy projektu, nebo veřejný, v tomto případě je dostupný pro všechny účastníky \gls{is}.

   Obsah je členěn do částí. Každá část tvoří samostatný celek, jenž se zakládá na určité šabloně (například šablona pro text, obrázek, graf), a může označovat jeden či několik úkolů za splněné. Seřazené části dávají výsledný obsah projektu, jenž se dá prohlížet online, případně zobrazit upravený pro tisk.

   \item[FR11 Projekt -- Snímky iterací]
   Aplikace poskytuje funkcionalitu vytvoření snímku určité iterace. Je to záznam aktuálního stavu všech částí obsahu, které splňují alespoň jeden z úkolů vybrané iterace. Tento záznam je dostupný pouze pro~čtení a existuje nezávisle na obsahu projektu. Snímek lze po vytvoření odevzdat pro ohodnocení, které spočívá v přiřazování bodů s volitelným komentářem k jednotlivým úkolům snímku. Dané hodnocení snímku se promítá do celkového přehledu iterací projektu. V případě, že existuje víc ohodnocených snímků jedné iterace, tak se bere nejnovější (nejpozději upravené). Existuje možnost změnit ohodnocení snímku, ale danou činnost může provádět pouze uživatel, jenž je původním hodnotitelem snímku.

   \item[FR12 Projekt -- Důvěryhodnost]
   Aplikace označuje projekt za důvěryhodný, pokud alespoň jeden z vedoucích je důvěryhodný. To se projevuje v podobě speciálního značení vedle názvu projektu.

   \item[FR13 Omezení obsahu \gls{is} pouze na projekty]
   Informační systém je pouze služba pro správu prací, nebude poskytovat žádnou jinou funkcionalitu pro uchovávaní dat s rychle stárnoucí hodnotou. Tím se zamezí nutnost neustálého udržování aplikace z hlediska obsahu.

\end{dl}


\subsection{Obecné požadavky}

\begin{dl}
   \item[NR00 Veřejné API]
   Aplikace nabízí otevřené \gls{api} pro vývojáře klientských aplikací.

   \item[NR01 Dokumentace]
   Součástí aplikace je vývojářská a uživatelská dokumentace.

   \item[NR02 Rozšiřitelnost]
   Aplikace bere v ohled budoucí rozvoj aplikace a zvyšující se nároky na server.

   \item[NR03 Optimalizace uživatelského rozhraní]
   Webové rozhraní aplikace je adaptované pro základní prohlížeče (Google Chrome, Mozilla Firefox, Apple Safari) a je responzivního nebo adaptivního typu optimalizované pro dvě velikosti obrazovek -- mobilních telefonů (pod 1~000~px) a desktopových počítačů (nad 1~000~px).

   \newpage
   \item[NR04 GDPR] 
   Aplikace splňuje požadavky \gls{gdpr}.

   \item[NR05 Jazykové verze] 
   Rozhraní aplikace bude dostupné v anglickém jazyce.
\end{dl}
